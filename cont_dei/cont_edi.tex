\documentclass{article}
\usepackage{amsmath}
\usepackage{graphicx}
\usepackage{float}
\usepackage{hyperref}
\usepackage[margin = 1in]{geometry}
\usepackage{xcolor}
\usepackage{subfig}
\usepackage{biblatex}
\addbibresource{References.bib}

\def\epo{\epsilon\rightarrow 0}
\def\lb{\left(}
\def\rb{\right)}
\def\ls{\left[ \vphantom]}
\def\rs{\right] }
\def\ep{\epsilon}

\title{Contributions towards Building a Diverse and Inclusive Field}
\author{}
\date{}

\begin{document}
\maketitle
\section*{Professional teaching and mentoring experiences}
Throughout my graduate studies at the University of Colorado Boulder, I have been involved with CU-Prime, a mentorship program that facillitates graduate student support of incoming undergraduates in Science, Technology, Engineering, and Math (STEM). 
The primary focus of the organization is to foster ``positive culture in STEM for those from typically underrepresented populations''\cite{cup}.
CU-Prime supports a graduate student talk series designed to reach a general undergraduate audience, as well as an introductory, single-credit course in which first year students get the opportunity to design and perform scientific research under the guidance of a graduate student. 
I have contributed multiple times to the talk series, and volunteered my time as a teaching assistant for the undergraduate course. 

I have taught numerous classes, however I found this course to be particularly challenging because the students were afforded so much flexibility in their own curriculum.
The students struggled initially; I think because scientific research feels unapproachable, with a huge barrier to entry, especially for those coming from \textcolor{red}{non-traditional backgrounds}. 
I struggled to help, because at first I felt similarly. Science is broad and hard, and it is super easy to feel lost.
But I was advised by senior grad students to simply support the ideas the students were most interested in, so that they had a sense of ownership of the project.
Unsurprisingly, it's good advice.
Admittedly, students are much more likely to remain interested in science, if the science is interesting to them. More importantly, it communicates that science is for anyone who has interesting questions to ask.

In my last year of college, a classmate confessed that she had once considered quitting our physics major because the work was difficult and-- she felt-- too hard for her. In hidsight, we both acknowledged that the classes were simply hard for everyone, but at the time, she said, she could not help wondering if the reason she was struggling was because she looked different from most everyone in the class.
%mentioned that we had all struggled, she responded that \textit{I} had probably not worried it was due to my own defficiencies, because when \textit{I} looked around the classroom, most everyone looked like me....
\textit{Science is hard.} When a student rightfully believes that they deserve to be involved, they are primed to understand that their struggles are due to this fact. 

When I teach, I make an effort to convey experiences from when I struggled with the content. When I have given the CU-Prime talks (example here \cite{jJ22}), I emphasize my own interests to illustrate why my research is a good fit for me, and how science could be a good fit for my audience.
\printbibliography
\end{document}

