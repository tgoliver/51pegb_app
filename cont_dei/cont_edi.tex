\documentclass[12pt]{article}
\usepackage{amsmath}
\usepackage{graphicx}
\usepackage{float}
\usepackage{hyperref}
\usepackage[margin = 1in]{geometry}
\usepackage{xcolor}
\usepackage{subfig}
\usepackage{biblatex}
\usepackage{fancyhdr}
\usepackage{sectsty}
\usepackage{titlesec}
\addbibresource{References.bib}

\def\epo{\epsilon\rightarrow 0}
\def\lb{\left(}
\def\rb{\right)}
\def\ls{\left[ \vphantom]}
\def\rs{\right] }
\def\ep{\epsilon}
\sectionfont{\fontsize{12}{15}\selectfont}
\subsectionfont{\fontsize{12}{15}\selectfont}


\author{}
\date{}

\begin{document}
\pagestyle{fancy}
\thispagestyle{fancy}
%... then configure it.
\fancyhf{} % clear all header fields
\fancyhead[L]{\textcolor{red}{Tobias Oliver\\
Contributions Towards Building a Diverse and Inclusive Field}}
\fancyfoot[R]{\thepage}
\titlespacing\section{0pt}{12pt plus 4pt minus 2pt}{0pt plus 2pt minus 2pt}
%\subsection*{Contributions towards Building a Diverse and Inclusive Field}
%\titlespacing\section{0pt}{12pt plus 4pt minus 2pt}{0pt plus 2pt minus 2pt}
The planetary sciences suffer from a severe lack of gender and racial diversity among faculty and graduate students\cite{kD20}, and an actionable step that is known to improve these conditions is to foster mentorship programs that serve underrepresented populations\cite{jH07}.
I have worked with CU-Prime, a mentorship program that facilitates graduate student support of incoming undergraduates in STEM. 
The program focuses primarily on culture building in STEM and providing positive spaces for typically underrepresented populations.
CU-Prime supports a graduate student talk series designed to reach a general undergraduate audience, as well as an introductory, single-credit course in which first year students get the opportunity to design and perform scientific research in small groups under the guidance of a graduate student. Although open to all first year students, enrollment preference is given to students from traditionally underrepresented backgrounds. 
I have contributed multiple times to the talk series and volunteered as a teaching assistant for the undergraduate course. 

%I have taught many introductory physics classes,
%however I found this course to be particularly challenging because the students were afforded so much flexibility in their own curriculum.
In the course, the students are required to propose and perform their own research project. 
The project does not have to be novel, but is certainly expected to go beyond the scope of a traditional, introductory physics lecture.
In the semester that I was involved, my job was to first help them generate an achievable proposal and then, over the remainder of the semester, assist in the setup and execution.
I struggled with the first task because the students were tentative and I had no idea what would be an appropriate topic for three students with only high school physics under their belt.
%The students are required to propose and perform a research project and, I think because scientific research has large barriers to entry, they often have a hard time coming up with a project they like, but that feels achievable. 
A senior graduate student advised that I simply support the ideas the students were most interested in. They would have a sense of ownership of the project,
%The students were much more likely to remain interested in science, if the science is interesting to them. 
but more importantly I would be communicating that science is for anyone who has interesting questions to ask.
I did not realize how important this was, until everyone was suddenly speaking up and getting excited.

%In my last year of college, a classmate confessed that she had once considered quitting our physics major because the work -- she felt-- was too hard for her. In hindsight, we both acknowledged that the classes were simply hard for everyone, but she told me that at the time she could not help wondering if the reason she was struggling was because she looked different from most everyone in the class.
%mentioned that we had all struggled, she responded that \textit{I} had probably not worried it was due to my own defficiencies, because when \textit{I} looked around the classroom, most everyone looked like me....
%When a student rightfully believes that they deserve to be involved, they are primed to understand that their struggles are due to this fact. 
I enjoy teaching and want it to be a part of my career.
I acknowledge that there are unique perspectives that mentors from minority backgrounds provide, and that there are many experiences that I cannot speak to, but I believe, and hope to convey, that the only barrier to science should be interest.

When I have performed public outreach, I try to make this a central message.
When I have given the CU-Prime talks, I emphasize my own interests to illustrate why my research is a good fit for me, and how science could be a good fit for my audience. 
I wish to continuing to prepare and present public research talks aimed at general science audiences.
I can commit to one talk per year, in part because I know how much effort it takes to communicate modern science effectively and engagingly.

I also think that interactive activities are an invaluable way to spark interest in science. 
I propose to facilitate hands on demonstrations of planetary fluid mechanics, again, about once a year.
Simple demonstrations of rotating fluids are interactive and fun.
Cheap experiments (often less than \$100) with rotating tanks of water can be easily set up and demonstrate a variety of weather and ocean phenomena. Students (and adults) can experiment by dropping food coloring into the tanks watch them swirl around. It is a fun way to learn about modern planetary science, and I would enjoy setting up and facilitating these demonstrations. 
%In my experience, even adults enjoy dropping food coloring into tanks of water and watching them swirl around. Furthermore, significant insights about weather and ocean circulation can be gained from these simple  experiments. I would contribute to these events and set up table-top rotating fluid experiments for students to play and learn with.

%The SPINlab at UC Los Angeles (the proposed host lab) already facilitates events like this and I would be excited to participate and contribute. The SPINlab also maintains a Youtube channel with outreach films. I would be thrilled to produce a series of videos discussing geophysical and astrophysical flows. 
\printbibliography
\end{document}

