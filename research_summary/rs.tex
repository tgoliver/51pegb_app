\documentclass[12pt]{article}
\usepackage{amsmath}
\usepackage{graphicx}
\usepackage{float}
\usepackage{hyperref}
\usepackage{xcolor}
\usepackage{subfig}
\usepackage[backend = biber,style = numeric,maxnames = 2,firstinits]{biblatex}
\usepackage[margin = 1in]{geometry}
\addbibresource{../instituion_rationale/UCSD/journal_abbreviations.bib}
\addbibresource{../proposal/References.bib}
\usepackage{sectsty}
\usepackage{titlesec}
\usepackage{fancyhdr}

\def\epo{\epsilon\rightarrow 0}
\def\lb{\left(}
\def\rb{\right)}
\def\ls{\left[ \vphantom]}
\def\rs{\right] }
\def\ep{\epsilon}

\sectionfont{\fontsize{12}{15}\selectfont}
\subsectionfont{\fontsize{12}{15}\selectfont}
\title{Summary of Previous and Current Research}
\author{Tobias Oliver}
\date{}

\AtEveryBibitem{% Clean up the bibtex rather than editing it
 \clearfield{isbn}
 \clearfield{doi}
 \clearfield{url}
 \clearfield{issn}
 }
 \titlespacing\subsection{0pt}{12pt plus 4pt minus 2pt}{0pt plus 2pt minus 2pt}

\begin{document}
\pagestyle{fancy}
\thispagestyle{fancy}
%... then configure it.
\fancyhf{} % clear all header fields
\fancyhead[L]{\textcolor{red}{Tobias Oliver\\
Summary of Previous and Current Research}}
\fancyfoot[R]{\thepage}
I study rotating convection with a primary focus on planetary interiors such as the liquid core of the Earth. 
My prior work has investigated turbulent length scales and force balances in rotating convection and has determined that eddy sizes in the core may be much smaller than previously thought. 
More recently, I have been studying how the liquid core exerts topographic torques onto the mantle of the Earth, which can affect changes in the Earth's length of day. My first publication on this topic provided theoretical scaling arguments for the magnitude of topographic torques. Ongoing work examines the effect of global scale topography on torques and convective dynamics. 
\subsection*{Length Scales in Rotating Convection \small{(Oliver et al. Phys. Rev. Fluids. 2023.)}}
%A turbulent fluid is characterized by chaotic fluctuations over a broad range of length scales. 
It is important to determine the largest length scale over which a turbulent fluid's behavior is correlated. 
%This quantity is often interpreted as the size of the largest eddies in the flow.
This scale is unknown for the liquid core, however its value is needed in order to better understand the geodynamo and core-mantle interactions. I performed numerical simulations of a dynamical model relevant to the Earth's liquid core, and investigated these length scales. Our findings suggest that previous work over-estimates this scale, and that the existing theory, based on force balances within the core, is incomplete. We propose a new theory which provides a new estimate for the eddy sizes and suggests that molecular viscosity plays a significant role in core flow. This behavior is uncommon in turbulence but is relevant to rotating planetary regimes.
\subsection*{Topographic Torques (Oliver et al. Comm. Ear. and Env. 2025, Oliver et al. \textit{in prep})}
Earth's length of day fluctuates in time, though the causes are not fully understood.
%By accounting for known processes such as atmospheric interactions and the secular increase of the lunar distance, an anomalous signal can be determined which must result from interactions between the solid mantle and the liquid core. 
One proposed mechanism for these variations are topographic torques, which involves the core fluid pushing on topography at the core-mantle interface and exerting a torque on the mantle. 
%We performed a novel set of numerical simulations and determined that this mechanism can produce a sufficiently large torque on the mantle to account for the anomalous signal.
I have performed the first systematic investigation of this problem through a set of direct numerical simulations in a non-spherical domain.
We provide a new theoretical framework to explain these torques and determine an asymptotic scaling to extrapolate our findings to planetary values.
Our results suggest that topographic torques are of sufficient magnitude to explain the anomalous length of day fluctuations. 
%Our study involved a novel numerical set-up in a non-spherical shell. To our knowledge, our work is the first to perform a forward simulation of rotating convection in a non-spherical shell domain. 
Ongoing work investigates more complicated topographies, including topographies motivated by seismic imaging of the core-mantle boundary.
\subsection*{Project Preparedness}
Rotating convection governs transport mechanisms in planetary interiors. It is expected to control the structure of icy moons, and understanding this process is one of my main goals.
%Rotating convection is an important topic in the study of planetary interiors, as I have shown in my research. 
My prior work on asymptotic scaling is particularly relevant because it allows for 
%I have focused on providing asymptotic scalings for these behaviors. 
extrapolation to planetary regimes. 
Given the shortage of observational data and computational limitations, this approach is necessary in order to make reliable predictions about icy moon oceans.
I have used a variety of numerical codes, and have significant experience building, modifying, and leveraging these codes for efficient use of computing resources. I am also able to relate numerical results to theory based on first principles physics.
\end{document}
