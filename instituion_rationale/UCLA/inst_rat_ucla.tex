\documentclass[12pt]{article}
\usepackage{amsmath}
\usepackage{graphicx}
\usepackage{float}
\usepackage{hyperref}
\usepackage{xcolor}
\usepackage{subfig}
\usepackage[backend = biber,style=numeric,maxnames =2,giveninits = true]{biblatex}
\addbibresource{../UCSD/journal_abbreviations.bib}
\addbibresource{../../proposal/References.bib}
\usepackage[margin = 1in]{geometry}
\usepackage{fancyhdr}
\usepackage{sectsty}
\usepackage{titlesec}

\def\epo{\epsilon\rightarrow 0}
\def\lb{\left(}
\def\rb{\right)}
\def\ls{\left[ \vphantom]}
\def\rs{\right] }
\def\ep{\epsilon}

\title{}
\author{}
\date{}
\AtEveryBibitem{% Clean up the bibtex rather than editing it
 \clearfield{isbn}
 \clearfield{doi}
 \clearfield{url}
 \clearfield{issn}
 }

\sectionfont{\fontsize{12}{15}\selectfont}
\subsectionfont{\fontsize{12}{15}\selectfont}
\begin{document}
\pagestyle{fancy}
\thispagestyle{fancy}
\titlespacing\subsection{0pt}{12pt plus 4pt minus 2pt}{0pt plus 2pt minus 2pt}

%... then configure it.
\fancyhf{} % clear all header fields
\fancyhead[L]{\textcolor{red}{Tobias Oliver\\
		Rationale for University of California Los Angeles\\
Proposed Mentor: Professor Jon Aurnou}}
\fancyfoot[R]{\thepage}
\setlength{\headheight}{25pt}
\subsection*{SPINlab}
The SPINlab is an experimental fluid dynamics laboratory in the Department of Earth, Planetary, and Space Sciences (EPSS). The lab focuses on investigating planetary flows in lab-sized rotating tanks of fluid. Dr. Aurnou oversees the lab and multiple experiments, which would provide excellent validation and inspiration for numerical work. In particular, a new, 1 m diameter paraboloidal fluid shell experiment is under way. With heterogeneous boundary heating and topography capabilities, the experiment will model complex processes within thin-shelled icy moons and is directly relevant to this proposal.

In rotating convection studies, it is very useful to run experiments and numerics concurrently. Although still falling short of geophysical regimes, experiments, such as the 40 cm tank experiment\cite{jA23} at SPINlab, are able to reach a more extreme parameter space than numerics. In a physical experiment, however, it is difficult to analyze flows because of the complex techniques required to track them. ``Numerical flows'' do not suffer from this challenge, since all flow data is stored in computer memory. By leveraging both approaches, experiment can inform numerics by suggesting where interesting behaviors may exist, and numerics can then be used to provide detailed, robust explanations.

EPSS is also the home department of Dr. Hao Cao, a planetary scientist who studies planetary flows and magnetic fields. Dr. Cao is a Co-Investigator on the JUICE mission, which will perform flyby's of Europa. In particular, Dr. Cao is involved with the J-MAG magnetometer, which will measure the magnetic field of Europa.
Although not a direct measurement of Europa's ocean dynamics, magnetic field measurements can be inverted (ie. \cite{nG15}) to hint at ocean processes, and the ability to collaborate with Dr. Cao would provide useful insight into relating the simulation results to actual measurements that JUICE (and \textit{Europa Clipper}) can make.

\textbf{UC Los Angeles would provide me access to the SPINlab where I would work with Dr. Aurnou. As a numericist, I would be able to collaborate with the experiment teams to better constrain the parameter regime that I search, as well as the manner in which I can extrapolate results to the icy-moons. Collaboration with Dr. Cao would enable better predictions about upcoming data from the JUICE mission.}
\subsection*{Outreach throgh SPINlab}
The SPINlab at UC Los Angeles (the proposed host lab) facilitates public outreach through interactive rotating tank experiments which I would be excited to contribute to. The lab also maintains a YouTube channel with outreach films. I would be thrilled to produce a series of videos discussing geophysical and astrophysical flows.
\printbibliography
\end{document}
