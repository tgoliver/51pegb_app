\documentclass{article}
\usepackage{amsmath}
\usepackage{graphicx}
\usepackage{float}
\usepackage{hyperref}
\usepackage{xcolor}
\usepackage{subfig}
\usepackage[backend = biber]{biblatex}
\addbibresource{../UCSD/journal_abbreviations.bib}
\addbibresource{../../proposal/References.bib}
\usepackage[margin = 1in]{geometry}

\def\epo{\epsilon\rightarrow 0}
\def\lb{\left(}
\def\rb{\right)}
\def\ls{\left[ \vphantom]}
\def\rs{\right] }
\def\ep{\epsilon}

\title{}
\author{}
\date{}

\begin{document}
\section*{Rationale for Host Institution: UC Los Angeles}
At UC Los Angeles, I would work with Dr. Jon Aurnou in the Department of Earth, Planetary, and Space Sciences (EPSS).
\subsection*{SPINlab at EPSS}
The SPINlab is an experimental fluid dynamics laboratory focused on investigating planetary flows in lab-sized rotating tanks of fluid. The lab is supervised by Dr. Aurnou. Dr. Aurnou oversees a number of different experiments, which would provide excellent validation and inspiration for numerical work. Most relevant, however, is a new, 1 m diameter paraboloidal fluid shell. With heterogrenous boundary heating and topography capabilities, the experiment will model complex processes within thin-shelled icy moons.

In rotating convection studies, it is very useful to run experiments and numerics concurrently. Although still falling short of geophysical regimes, experiments, such as the 40 cm tank experiment\cite{jA23} at SPINlab, are able to reach a more extreme parameter space than numerics, especially in the context of exploring the ``transitional'' regime thought to be relevant to icy moons (see proposal). On the contrary, it is difficult to analyze physical flows, because of the complex techniques required to track them. Numerically this is not a challenge, since all flow data is stored in memory. By leveraging both approaches, experiment can inform numerics by suggesting where interesting behaviors may exist, and numerics can then be used to provide detailed, robust explanations.

EPSS is also the home department of Dr. Hao Cao, a planetary scientist who studies planetary flows and magnetic fields. Dr. Cao is a Co-Investigator on the JUICE mission, which will perform flyby's of Europa. In particular, Dr. Cao is involved with J-MAG magnetometer, which will measure the magnetic field of Europa.
Although not a direct measurement of Europa's ocean dynamics, magnetic field measurements can be inverted (ie. \cite{nG15,dJ88}) to hint at ocean processes, and the ability to collaborate with Dr. Cao would provide useful insight into relating the simulation results to actual measurements that JUICE (and \textit{Europa Clipper}) can make.

\textbf{UC Los Angeles would provide me access to the SPINlab where I would work on Jon Aurnou. As a numericist, I would be able to collaborate with the experiment teams to better constrain the parameter regime that I search, as well as the manner in which I can extrapolate results to the icy-moons. Collaboration with Dr. Cao would enable better predictions about upcoming data from the JUICE mission.}
\printbibliography
\end{document}
